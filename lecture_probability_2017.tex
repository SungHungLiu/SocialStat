\documentclass{beamer}
\usepackage{amsmath, amsfonts, epsfig, xspace}
\usepackage{algorithm,algorithmic, graphicx}
\usepackage{pstricks,pst-node}
\usepackage{multimedia}
\usepackage[normal,tight,center]{subfigure}
\setlength{\subfigcapskip}{-.1em}
\usepackage{beamerthemesplit}
\usepackage{xcolor}
\usepackage{multirow}
\usepackage{IEEEtrantools}
\newcommand{\non}{\nonumber}
\setbeamertemplate{enumerate items}[circle]
\setbeamertemplate{navigation symbols}{}
\usepackage{fontspec}  %加這個就可以設定字體
\usepackage{xeCJK}       %讓中英文字體分開設置
\setCJKmainfont[ItalicFont=王漢宗中仿宋繁, BoldFont=標楷體]{王漢宗細圓體繁}
%\newfontfamily{\K}{BiauKai}
\setmonofont[Scale=0.8]{Courier New} % 等寬字型
%\setCJKmainfont{細明體} %設定中文為系統上的字型,而英文不去更動,使用原TeX字型
\XeTeXlinebreaklocale "zh"             %這兩行一定要加,中文才能自動換行
\XeTeXlinebreakskip = 0pt plus 1pt     %這兩行一定要加,中文才能自動換行
%\usetheme{lankton-keynote}
\usetheme{Madrid}

\author[蔡佳泓]{蔡佳泓}

%\title[Statistical Methods for Social Sciences\hspace{.1em}\insertframenumber/\inserttotalframenumber]{Probability Theory}

\title[Statistical Methods for Social Sciences]{機率簡介}

\date{2017/2/21} %leave out for today's date to be insterted

\institute[]{國立政治大學東亞所暨選舉研究中心}

\begin{document}
\maketitle
\tableofcontents
\section{樣本空間}
\begin{frame}\frametitle{樣本空間}
\begin{itemize}
\item 將一個實驗中所有可能出現的結果 (outcomes) 所形成的集合稱為一樣本空間$S$。
\item 例如德州奧斯汀的人很自豪地說世界上只有三種人: 第一是不知道有奧斯汀這個地方的人,第二是知道奧斯汀但恨不能來此定居的人,第三是已經住在這裡的人。假如這個說法為真,我們隨便問一個美國人,那麼他應該屬於$S=\{$不知道有奧斯汀, 知道奧斯汀但恨不能來此定居, 已經住在這裡的人$\}$其中之一。
\item 由於每一種可能的結果皆屬於樣本空間,故亦可稱為元素或樣本點。
\item 事件 (event) 為結果所形成的集合,事件為 $S$ 的子集,以大寫字母 ($A$, $B$, $C$,\ldots) 表示,可分為以下兩種類型:
\begin{enumerate}
\item 事件中只有包含一個元素稱作簡單事件 (simple event, 亦稱為樣本點) 例如投擲一顆骰子得點數 3。
\item 事件中包含兩個以上的元素稱作複合事件 (compound event),例如投擲兩顆骰子得點數和為 3 或 5 或 7。
\end{enumerate}
\item 若 A, B 兩事件滿足 $A \cap B = \emptyset $時, 則稱此$A$ 事件與 $B$ 事件為互斥事件 (disjoint events)。
\end{itemize}
\end{frame}
\begin{frame}\frametitle{實例}
\begin{itemize}
\item 丟一枚硬幣,樣本空間$S=\{$正面,反面$\}$,出現正面的事件$E=\{$正面$\}$
\item 丟一個六面的骰子,樣本空間$S=\{1,2,3,4,5,6\}$,出現偶數的事件$E=\{2,4,6\}$
\item 丟一枚硬幣兩次,樣本空間$S=\{HH, HT, TH, TT\}$,第一次丟擲出現正面的事件為$E=\{HH, HT\}$
\item 丟一個骰子兩次,樣本空間$S=\{(i,j):i,j=1,2,\ldots,6\}$,共有36個元素。兩次骰子點數總和為10為事件之一。
\end{itemize}
\end{frame}
\begin{frame}\frametitle{機率公設}
\begin{theorem}
\begin{enumerate}
\item 對於任意事件 $E$ 皆滿足:$P(E) \geq 0$。(Non-negativity)
\item 樣本空間 $S$ 的機率等於 1,寫作 $P(S) = 1$。(Normalization)
\item 如果 $A_{1}$,$A_{2}$,$A_{3}$\ldots ,為一組有限或者可數的無限的事件且彼此為互斥事件,則這些事件所聯集的機率等於個別事件的機率和等於\newline
$P(A_{1} \cup A_{2} \cup A_{3} ∪ · · ·) = P(A_{1}) + P(A_{2}) + P(A_{3}) + \cdots$ (Additivity)
\end{enumerate}
\end{theorem}
\end{frame}

\section{Experiments, Sample Spaces, Events and Probability Law}  % add these to see outline in slides
%\begin{frame} \frametitle{樣本空間}
%  \begin{itemize}
%  \item 離散: $\Omega=\{1,2,3,4,5,6\}$
%  \item 連續: $\Omega=\{1,2,3,\cdots \cdots,\infty \}$
%  \item 事件 A 定義為其中的奇數 $\Omega=\{1,2,3,4,5,6\}$.\\ A=$\{1,3,5 \}$ and $P(A)=\frac{1}%{6}+\frac{1}{6}+\frac{1}{6}=0.5$
%  \end{itemize}
%\end{frame}

\begin{frame}\frametitle{實例}
\begin{itemize}
\item Example: rolling two six-sided dice and recording the sum. What is $P(S=3)$? The sample space and probabilities of each event is :\\
\end{itemize}
\begin{table}[ht]
\centering
\footnotesize
\begin{tabular}{|l|l|l|l|l|l|l|c|}
\hline
sum&1&2&3&4&5&6&Probability\\
\hline
2&$\{1,1\}$&&&&&&$\frac{1}{36}$\\
\hline
3&$\{1,2\}$&$\{2,1\}$&&&&&$\frac{2}{36}$\\
\hline
4&$\{1,3\}$&$\{2,2\}$&$\{3,1\}$&&&&$\frac{3}{36}$\\
\hline
5&$\{1,4\}$&$\{4,1\}$&$\{2,3\}$&$\{3,2\}$&&&$\frac{4}{36}$\\
\hline
6&$\{1,5\}$&$\{5,1\}$&$\{3,3\}$&$\{2,4\}$&$\{4,2\}$&&$\frac{5}{36}$\\
\hline
7&$\{1,6\}$&$\{6,1\}$&$\{3,4\}$&$\{4,3\}$&$\{2,5\}$&$\{5,2\}$&$\frac{6}{36}$\\
\hline
8&$\{2,6\}$&$\{6,2\}$&$\{3,5\}$&$\{5,3\}$&$\{4,4\}$&&$\frac{5}{36}$\\
\hline
9&$\{3,6\}$&$\{6,3\}$&$\{4,5\}$&$\{5,4\}$&&&$\frac{4}{36}$\\
\hline
10&$\{4,6\}$&$\{6,4\}$&$\{5,5\}$&&&&$\frac{3}{36}$\\
\hline
11&$\{5,6\}$&$\{6,5\}$&&&&&$\frac{2}{36}$\\
\hline
12&$\{6,6\}$&&&&&&$\frac{1}{36}$\\
\hline
\end{tabular}
\end{table}
\end{frame}
%\begin{frame}\frametitle{Probability Law}
%   \begin{itemize}
 % \item Non-negativity: $P(A)\geq 1 $
%  \item Normalization: $P(\Omega)=1$
%  \item Additivity: When $A_{1}$, $A_{2}$ are mutually exclusive, the probability of $A_{1}$ or $A_{2}$ occurring is $P(A_{1})+P(A_{2})$.
%  \end{itemize}
%\end{frame}
\section{Probability Rules}
\begin{frame}
  \frametitle{Probability Rules}
    \begin{itemize}
     \item $P(\emptyset)=0$
     \item $0\leq P(E)\leq 1$
  \item If $A \subseteq B$ then $P(A)\leq P(B)$
  \item If $a$ is 3, and $A=\{$all odd numbers$\}$, $a \in A$
  \item Subset: If $T=\{1,3,5\}$, $T \subset A$
  \item Union: If $B=\{1,2,\cdots,10\} $, $ B\cup T=\{1,2,3,\cdots, 10\} $
  \item Intersection: $ B\cap T=\{1,3,5\} $
  \item Empty set: If $S=\{2\}  $, $S \cap T=\emptyset$
 
  \item Complement (餘集): If $T$ relative to $B$, $T^c=\{1,2,\cdots,10\}\diagdown\{1,3,5\}=\{2,4,6,\cdots,10\}$
  \end{itemize}
\end{frame}

\begin{frame}\frametitle{Operations on sets}
    \begin{itemize}

  \item $P(A\cup B)=P(A)+P(B)-P(A\cap B)$
  \item $(A\cup B)^C=A^c\cap B^c  $ (De Morgan's Law 1)\\
  Proof: \begin{IEEEeqnarray}{rl}
  A^c\cap B^c  & =(1-P(A))\times (1-P(B)) \non\\
&   =1-P(A)-P(B)+P(A)\times P(B) \non\\
&   =1-(P(A)+P(B)-P(A\cap B))\non \\
&   =(A \cup B)^c \non
\end{IEEEeqnarray}
\item $(A\cap B)^C=A^c\cup B^c  $ (De Morgan's Law 2)
  \end{itemize} 
\end{frame}
\begin{frame}\frametitle{應用 De Morgan's Law}
\begin{itemize}
\item 有一個400戶的社區,160戶獨棟,260 戶為有一個以上對外窗,120戶為獨棟而且有一個以上對外窗。300 戶為獨棟或者是有一個以上對外窗。隨機抽一戶,請問這一戶不是獨棟或沒有一個以上對外窗的機率多高?
\item 假設A代表獨棟的事件,B代表有一個以上對外窗的事件,\\
\begin{IEEEeqnarray}{rl}
P((A\cup B)^C) & =P( A^c\cap B^c) \non \\
 & = 1-(\frac{160}{400}+\frac{260}{400}-\frac{120}{400}) \non \\
 & = 1-\frac{300}{400} \non \\
 & = 0.25 \non
\end{IEEEeqnarray}
  \end{itemize} 
\end{frame}


\subsection{聯合機率 }
\begin{frame}\frametitle{定義}
\begin{itemize}
\item 兩個事件同時發生的機率,或者是兩個事件交集的機率,可表示為 $p(A,B)$ 或者是 $p(A\cap B)$.
\item 例如: 一張撲克牌,同時為紅色且為四點的機率為$p(four, red)=2/52=1/26$
\item 民調顯示,有17\%的美國人同時養狗與養貓,可表示為$p(dog\cap cat)=0.17$
\end{itemize}
\end{frame}
\begin{frame}\frametitle{獨立事件 (Independent Events)}
\begin{itemize}
\item 如果事件 $A$ 發生時與事件 $B$ 沒有關係,或者是當 $P(A,B)=P(A)\cdot P(B)$, 也就是 $P(A|B)\cdot P(B)=P(B)$,這兩個事件稱為獨立事件。
\item 擲一顆骰子兩次,第一次得到6的機率與第二次得到6的機率相互獨立。
\item 抽一張撲克牌,放回去之後再抽一次,第一次抽到紅心的機率與第二次抽到紅心的機率相互獨立。
\item $P(A\cap B)=P(A,B)=P(A)\cdot P(B)$ 
\end{itemize}
\end{frame}

\subsection{互斥事件 (Disjoint Events)}
\begin{frame}
\begin{itemize}
\item 兩個事件不可能同時發生,稱為互斥事件。兩個互斥事件的機率 $P(A\cup B)=P(A)+P(B)$
\item 例如,一個學生不可能同時為大一學生,也是大四學生,所以是互斥事件。但是一個學生可以同時為大一學生
也是男生。
\item 例如: $C=\{1,3,5\}, D=\{2,4,6\}$. $C$ 與 $D$ 事件為互斥。 $C\cap B=\emptyset$. $P(C,D)=0$
\end{itemize}
\end{frame}
\section{課堂問答}
\begin{frame}{課堂問答}
\begin{enumerate}
\item 請問氣溫與看病人數是否為獨立事件? 還是互斥事件? 如何檢測?
\item 請問抽到紅色的牌與抽到有臉的牌是否為獨立事件?\pause
\begin{itemize}
\item 抽到紅色牌的機率$P(Red)=\frac{1}{2}$ \pause
\item 抽到有臉的牌的機率$P(Face)=\frac{12}{52}=\frac{3}{13}$ \pause
\item 抽到紅色且有臉的牌的機率$P(R,F)=\frac{6}{52}=\frac{3}{26}$\pause
\item $\frac{1}{2}\cdot\frac{3}{13}=\frac{3}{26}$
\end{itemize}
\end{enumerate}
\end{frame}

\section{邊際機率(Marginal Probability)}
\begin{frame}\frametitle{邊際機率 (Marginal Probability)}
\begin{itemize}
\item 在兩個或兩個以上的樣本空間中,只考慮某一個條件成立所發生的機率。
%\item $P(A)$ is not conditional on the probability of other events.
\item 例子: 有一個樣本空間 $S=\{\textcolor{red}{1,2,3,4}, \textcolor{blue}{5,6},7,8,9,10\}$ . $P(A)=P(x={\rm black}\}=0.4$. $P(B)=P\{x\geq 6\}=0.5$.\\
\vspace{1cm}
\item 假設抽出50位中國的重要人物,並且調查政治與企業的資歷,得到結果如下:
\begin{table}[ht]
\begin{tabular}{ r | c | c | }
\multicolumn{1}{r}{}
 &  \multicolumn{1}{c}{任職央企($A_{1}$)}
 & \multicolumn{1}{c}{任職地方國企($A_{2}$)} \\
\cline{2-3}
 政治局委員($A_{3}$) & 12 & 14 \\
\cline{2-3}
 非政治局委員($A_{4}$) & 11 & 13 \\
\cline{2-3}
\end{tabular}
\end{table}
\item 曾任職央企的機率:$P(A_{1})=P(A_{1}\cap A_{3}) + P(A_{1}\cap A_{4}) = 0.24 + 0.22 = 0.46$
\end{itemize}
\end{frame}
\section{條件機率(Conditional Probability)}
\begin{frame}\frametitle{條件機率(Conditional Probability)}
\begin{itemize}
\item The probability of event A when B is true (or occurs). It is denoted as $P(A|B)$. The probability of $A$ occurring is conditional on $B$ having occurred.
\item Manipulation: $P(A|B)=\frac{P(A\cap B)}{P(B)}$. The joint probability of A and B divided by the marginal probability of B.
\item $P(A\cap B)=P(A|B)\cdot P(B)=P(B\cap A)=P(B|A)\cdot P(A)$
\end{itemize}
\end{frame}

\begin{frame}{總機率(Total Probability)}
\begin{itemize}
\item If there are mutually exclusive events in the sample space, such as odd and even numbers of rolling a dice, the intersection between each of them and the other event represents the partition of the other event.
\item Example: People are either male or female. Suppose 55\% of people are male, and it is found that smoking men are 40\% of people and non-smoking are 15\%. 
\begin{table}[h]
\begin{tabular}{c|l|l|c}
&smoking&non-smoking& total\\
\hline
male&0.4&0.15&0.55\\
\hline
female&0&0.45&0.45\\
\hline
total&0.4&0.6&1\\
\end{tabular}
\end{table}
\item $P({\rm male})=P({\rm male, smoking})+P({\rm male, non-smoking})$
\end{itemize}
  \begin{alertblock}{Total Probability}
  $P(A)=P(A\cap B)+P(A\cap B^C)$
  \end{alertblock}
\end{frame}
\begin{frame}\frametitle{互賴事件 (Dependent Events)}
\begin{itemize} 
\item 互賴事件指的是A事件的機率會影響B事件的機率。可以定義為兩個事件的聯合機率不等於相乘。或者是條件機率不等於邊際機率。
\item 範例:請問以下的資料中,年齡與學術專長是否互相影響?
\vspace{.5cm}
\begin{table}[ht]
\begin{tabular}{ r | c | c |  c}
\multicolumn{1}{r}{}
 &  \multicolumn{1}{c}{60後($A_{1}$)}
 & \multicolumn{1}{c}{非60後($A_{2}$)} & \\
\cline{2-3}
 人文社科背景($A_{3}$) & 0.30 & 0.26 & 0.56\\
\cline{2-3}
 理工科背景($A_{4}$) & 0.16 & 0.28  & 0.44\\
\cline{2-3}
\multicolumn{1}{r}{}
 &  \multicolumn{1}{c}{0.46} & \multicolumn{1}{c}{0.54} & 1 \\
\end{tabular}
\end{table}

\item $P(A_{3}|A_{1})=\frac{P(A_{1}\cap A_{3})}{P(A_{1})}=\frac{0.3}{0.46}=0.65$
\item $P(A_{3})$=0.56
\item 因為條件機率不等於邊際機率,所以是否為60後出生的機率與所學領域的機率兩者為互賴事件。
\end{itemize}
\end{frame}

\begin{frame}\frametitle{Multiple Events}
假設有事件$B$以及互斥事件 $A_{1}$, $A_{2}$, $A_{3}$ 
\begin {IEEEeqnarray}{rl}
P(B)&=P(B\cap A_{1})+P(B\cap A_{2})+P(B\cap A_{3})\non\\ 
&=P(A_{1})\cdot P(B|A_{1})+P(A_{2})\cdot P(B|A_{2})+P(B)\cdot P(B|A_{3})\non
\end {IEEEeqnarray}
\begin{itemize}
\item 例如:從52張撲克牌中以抽出不放回的方式抽出兩張牌,第2張牌是皇后的機率為何?
\item 假設$P(E)$代表第2張牌是皇后的機率,$P(A)$是第一張牌為皇后的機率,$P(A')$是第一張牌不是皇后的機率
\item $P(A)=\frac{4}{52}$, $P(A')=\frac{48}{52}$, $P(E|A)=\frac{3}{51}$, $P(E|A')=\frac{4}{51}$
\end{itemize}
\begin {IEEEeqnarray}{rl}
P(E)&=P(E\cap A)+P(E\cap A')\non\\ 
& = P(A)\cdot P(E|A)+P(B)\cdot P(E|B)\non\\ 
& = \frac{4}{52}\cdot\frac{3}{51}+\frac{48}{52}\cdot\frac{4}{51} \non\\ 
& = \frac{1}{13} \non
\end {IEEEeqnarray}

\end{frame}
\section{貝氏定理}
\begin{frame}\frametitle{貝氏定理}
\begin{itemize}
\item 運用已經的資訊更新對於未知參數的估計。\textbf{未知參數是隨機變數,不是(無法觀察的)固定的值。}
\item 例如:\phi = 3.141$\cdots$ ,請問第77個數字是不是3?按照機率理論,是3的機率為0或1,但是按照貝氏定理,如果已知0$\sim$9都有可能出現,那麼第77個數字是3的機率是1/10,。
\item $P(A|B)=\frac{P(A\cap B)}{P(B)}=\frac{P(B\cap A)}{P(B)}=\frac{P(B|A)\cdot P(A)}{P(B)}$
\item 根據全機率原理,一個事件的機率等於它與其他事件的交集的總和。例如事件 $A$ 可以分成 $A$與$A^c$, 事件 $B$ 與 $A$ 的條件機率可以寫成如下:
\end{itemize}
\begin{alertblock}{Bayes Law}
$P(A|B)=\dfrac{P(B|A)\cdot P(A)}{P(B)}=\frac{P(B|A)\cdot P(A)}{P(B|A)\cdot P(A)+P(B|A^c)\cdot P(A^c)}$
\end{alertblock}
\end{frame}

\subsection{實例}
\begin{frame}\frametitle{貝氏定理實例1}
\begin{itemize}
\item 已知 0.1 \% 的人口罹患肺結核,而且已知 X 光檢查得出 90 \% 有肺結核的病患,但是有 10 \% 檢查錯誤。而對沒有患肺結核的人,X光檢查結果是99 \% 正確,但是 1 \% 誤判成沒有結核病。假設有人接受肺結核檢查而且結果顯示有肺結核,請問他確實得到肺結核的機率是多少? 
\item 以 $P(V)$ 代表有結核病的機率, $P(V)=0.001$,$P(V')=0.999$。 $P(X)$ 表示檢查結果為有結核病的機率。根據上述,$P(X|V)=0.9$,$P(X'|V)=0.1$,$P(X'|V')=0.99$,$P(X|V')=0.01$
\end{itemize}
\end{frame}
\begin{frame}\frametitle{貝氏定理實例1}
\begin{itemize}
\item 根據貝氏定理,$P(V|X)$可表示如下:

\begin{IEEEeqnarray}{rl}
P(V|X) & =\frac{P(V\cap X)}{P(X)} =\frac{P(X\cap V)}{P(X)} \non\\
& =\frac{P(V)\cdot P(X|V)}{P(X)\cdot P(X|V)+P(V')\cdot P(X|V')}=\frac{0.001\cdot 0.9}{0.001\cdot 0.9+0.999\cdot 0.01} \non\\
& =\frac{0.0009}{0.0009+0.009}\approx 0.082 \non
\end{IEEEeqnarray}
\item 有結核病的機率原本為0.001,而增加 X 光檢查之後,機率上升到0.082
\end{itemize}
\end{frame}
\begin{frame}{表格}
\begin{table}
\begin{tabular}{c|c|c|c}
& $P(X)$ & $P(X')$& \\
\hline
$P(V)$& $P(V\cap X)$&$P(V\cap X')$&$P(V)$\\
& 0.0009 & 0.0001&0.001\\
\hline
$P(V')$&$P(V'\cap X)$&$P(V'\cap X')$&$P(V')$ \\
 & 0.0099 & 0.9890 & 0.999\\
\hline
&$P(X)$&$P(X')$&\\
total&0.01089&0.98911&1\\
\end{tabular}
\end{table}
\begin{block}{Bayes Law}
$P(E_{k}|E)=\frac{P(E_{k})\cdot P(E|E_{k})}{P(E_{k})\cdot P(E|E_{k})+P(\sim E_{k})\cdot P(E|\sim E_{k})}$\\
$1\leq k\leq n$
\end{block}
\end{frame}
\begin{frame}{實例2}
\begin{itemize}
\item 過去研究發現,論文造假的機率為0.001。假如論文造假,且被檢舉的機率為0.05,論文沒有造假,卻被檢舉(或質疑)造假的機率為0.005。請問被檢舉的論文實際上造假機率是多少?
\item 以 $P(F)$ 代表論文造假的機率, $P(F)=0.001$。 $P(E)$ 表示檢舉結果為造假的機率。根據上述,$P(E|F)=0.05$,$P(E|F')=0.005$
\item 根據貝氏定理,$P(F|E)$可表示如下:
\begin{IEEEeqnarray}{rl}
P(F|E) & =\frac{P(F\cap E)}{P(E)}  \non\\
& =\frac{P(F)\cdot P(E|F)}{P(E)\cdot P(E|F)+P(F')\cdot P(E|F')}=\frac{0.001\cdot 0.05}{0.001\cdot 0.05+0.999\cdot 0.005} \non\\
& =\frac{0.0005}{0.0005+0.004995}\non\\
& =\frac{0.0005}{0.005495} \approx 0.091 \non
\end{IEEEeqnarray}
\item 造假的機率原本為0.001,而增加檢舉之後,機率上升到0.091
\end{itemize}
\end{frame}
\section{Exercises}
\begin{frame}\frametitle{Exercises}
\setbeamertemplate{enumerate item}{%
  \usebeamercolor[bg]{item projected}%
  \raisebox{-.5pt}{\colorbox{bg}{\color{fg}\footnotesize\insertenumlabel}}%
}

\begin{enumerate}
\small
\item Please write down the sample space of tossing three coins.
\item Roll the dice twice and record whether each roll is even or odd. Please write down the sample space $\Omega$ .
\item $P(A)$ and $P(B)$ denotes the probability of getting an odd and even number by rolling a dice, respectively. Are they independent? 
\item It is estimated that 40\% of Kaohsiung residents visit National Sun-Yat-Sen University last year.  Three Kaohsiung residents are selected at random.  What is the likelihood that all of them visited the university? 
\item Suppose 99\% of drivers tackle their seat belts but 1\% of drivers fail to do so. Of drivers who follow the rule, 0.5\% are stopped by the police but 99.5\% are not. Of drivers who fail to use their seat belts, 20\% are detected but 80\% do not get caught. Suppose the police stops a car randomly, what is the probability that the driver gets the fine of violating the rule?
\end{enumerate}

\end{frame}
\end{document}
